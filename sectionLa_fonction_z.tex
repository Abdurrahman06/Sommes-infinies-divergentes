\section{La fonction $\zeta(z)$ de Riemann}
La fonction $\zeta$ est d'une extrême importance en théorie des nombres, en physique mathématique et fait l'objet d'une conjecture qui est une des plus importante des mathématiques, non résolue depuis plus de 150 ans. Pour un nombre réel $x$, $\zeta(x)$ a été définie par L. Euler au 18\ieme~siècle:

\begin{equation}
\zeta(x) = \sum_{n=1}^\infty \frac{1}{n^x},
\end{equation}

En passant, cette série est importante pour la théorie des nombres car Euler a montré que

\begin{equation}
\zeta(s) = \prod_{p \text{ premier}}\frac{1}{1-p^{-s}}.
\end{equation}

Cette somme n'est définie dans $\Rset$ que pour $x > 1$. B. Riemann a proposé dans~\cite{Riemann1859} la continuation analytique de cette fonction, une manière pour la calculer partout dans $\Cset$, et la fameuse {\em hypothèse de Riemann}, qui fait l'objet d'un prix d'un million de dollars~\cite{RiemannHyp}, et qui dit que tous les \og zéros non triviaux \fg de $\zeta(z)$ sont situés sur la droite $z(t) = it + \frac{1}{2}$ (autrement dit, la droite de tous les nombres complexes qui ont pour partie réelle $\frac{1}{2}$).

\subsection{Calcul de $\zeta(-1)$}
On peut effectivement calculer $\zeta(z)$ pour tout nombre complexe $z \neq 1$. En particulier, le calcul suivant est également dû à Euler. Reprennons l'expression de $S(x)$, et dérivons la :

\begin{align}
S(x) = \frac{1}{1-x} &= 1 + x + x^2 + x^3 + \ldots\\
\frac{\diff S(x)}{\diff x} = \frac{1}{(1-x)^2} &= 0 + 1 + 2x + 3x^2 + \ldots
\end{align}

En subsituant $x = -1$, on a:

\begin{equation}
\frac{1}{4} = 1 - 2 + 3 - 4 + 5 + \ldots = \sum_{n=1}^\infty n(-1)^{(n-1)} 
\end{equation}
Un autre résultat inattendu. Maintenant, revenons à l'expression de $\zeta(z)$.

\begin{equation}
\begin{array}{rlllllllll}
\zeta(z)             &= 1^{-z} &+& 2^{-z} &+& 3^{-z} &+& 4^{-z} &+& \ldots\\
2^{-z}\zeta(z)       &=        & & 2^{-z} & &        &+& 4^{-z} &+& \ldots\\
\zeta(z)(1-2.2^{-z}) &= 1^{-z} &-& 2^{-z} &+& 3^{-z} &-& 4^{-z} &+& \ldots
\end{array} 
\end{equation}

En substituant $x=-1$, on obtient~:

\begin{equation}
\zeta(-1)(-3) = 1 - 2 + 3 - 4 + \ldots = \sum_{n=1}^\infty n(-1)^{(n-1)} = \frac{1}{4}
\end{equation}

donc~:

\begin{equation}
\zeta(-1) = -\frac{1}{12},
\end{equation}

donc, {\em dans ce sens}, on peut écrire:

\begin{equation}
\zeta(-1) = \sum_{n=1}^{+\infty} \frac{1}{n^{-1}} = \sum_{n=1}^{+\infty} n = -\frac{1}{12}.
\end{equation}

Cette dernière relation explique l'équation~\eqref{eq.zeta-1}. Un calcul similaire donne $\forall n \in \Nset, \zeta(-2n) = \zeta(-2) =  0$. Ceux-ci sont les zéros dits \og triviaux \fg de la fonction $\zeta$.

Un point important est qu'il ne faut pas confondre l'opération de sommation infinie avec la notation de sommation infinie. L'opération de sommation infinie (effectivement effectuer $1+2+3+\ldots$, sur papier ou calculatrice) diverge. La {\em notation} $\sum_{n=1}^{\infty} n$ est une autre entité, qui a un sens dans le cas du calcul de la variable complexe.

\subsection{Autres sommations}
Il est possible d'effectuer la sommation de~\eqref{eq.zeta-1} de plusieurs autres fa\c{c}ons, par exemple~\cite{Ramanujan} et heureusement toutes donnnent le même résultat.