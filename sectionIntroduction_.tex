\section{Introduction}

Une somme infinie divergente apparaît très tôt dans le traitement mathématique de la théorie des cordes. Par exemple, on trouve dans~\cite{stringtheory98}, chapitre 1, p. 22, eq. (1.3.32)

\begin{equation}
\sum_{n=1}^{\infty} n \rightarrow -\frac{1}{12},\label{eq.zeta-1},
\end{equation}
ce que l'auteur appelle un « odd result ». D'où vient cette somme étrange ? En fait c'est un raccourci extrême, qui provient d'une technique dite de « continuation analytique ».